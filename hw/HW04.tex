\documentclass[12pt]{article}
\usepackage[left=2cm,right=2cm,top=2cm,bottom=2cm,bindingoffset=0cm]{geometry}
\usepackage[utf8x]{inputenc}
\usepackage[english,russian]{babel}
\usepackage{cmap}
\usepackage{amssymb}
\usepackage{amsmath}
\usepackage{url}
\usepackage{pifont}
\usepackage{tikz}
\usepackage{verbatim}

\usetikzlibrary{shapes,arrows}
\usetikzlibrary{positioning,automata}
\tikzset{every state/.style={minimum size=0.2cm},
initial text={}
}


\newenvironment{myauto}[1][3]
{
  \begin{center}
    \begin{tikzpicture}[> = stealth,node distance=#1cm, on grid, very thick]
}
{
    \end{tikzpicture}
  \end{center}
}


\begin{document}
\begin{center} {\LARGE Формальные языки} \end{center}

\begin{center} \Large Домашнее задание 4\\Дмитрий Орехов \end{center}
\bigskip

\section{}
Доказать или опровергнуть свойство регулярных выражений:
  \[
    \forall p, q \text{ --- регулярные выражения}: (p \mid q)^* = p^*(qp^*)^*
  \]

\section{}
Доказать или опровергнуть свойство регулярных выражений:
  \[
    \forall p, q \text{ --- регулярные выражения}: (p q)^* p = p (q p)^*
  \]

\section{}
Доказать или опровергнуть свойство регулярных выражений:
  \[
    \forall p, q \text{ --- регулярные выражения}: (p q)^* = p^* q^*\\
  \]
  Данное свойство не выполняется. Рассмотрим некоторое слово $w$, принимаеое выражением $q$, но не принимаемое выражением $p$. Выражение $p^*q^*$ принимает слово $w$, так как мы можем полностью пропустить $p$. Выражение $(pq)*$ не позволяет пропустить $p$ и не принимает слово $w$.\\
  Таким образом, два выражение не эквивалентны.

\section{}
Для регулярного выражения:
   \[ (a \mid b)^+ (aa \mid bb \mid abab \mid baba)^* (a \mid b)^+\]
  Построить эквивалентные:
  \begin{enumerate}
    \item Недетерминированный конечный автомат
    \item Недетерминированный конечный автомат без $\varepsilon$-переходов
    \item Минимальный полный детерминированный конечный автомат
  \end{enumerate}

a. Нарисую отдельно автомат для выражения $(a|b)^+$, далее буду обозначать данный автомат как $P$: \\
\begin{myauto}
  \node[state,initial] (q_0) {A};
  \node[state]         (q_1) [above right of=q_0] {B};
  \node[state]         (q_2) [below right of=q_0] {C};
  \node[state]         (q_3) [right of=q_1]       {D};
  \node[state]         (q_4) [right of=q_2]       {E};
  \node[state]         (q_5) [below right of=q_3] {F};
  \node[state,accepting]         (q_6) [below of=q_0]       {G};

  \path[->] (q_0) edge node [above] {$\epsilon$} (q_1)
                  edge node [below] {$\epsilon$} (q_2)
            (q_1) edge node [above] {$a$}        (q_3)
            (q_2) edge node [below] {$b$} (q_4)
            (q_3) edge node [above] {$\epsilon$} (q_5)
            (q_4) edge node [below] {$\epsilon$} (q_5)
            (q_5) edge [bend left=50] node [below] {$\epsilon$} (q_6)
            (q_6) edge node [left] {$\epsilon$} (q_0);
\end{myauto}\\
Автомат для выражения $(aa|bb|abab|baba)^*$, далее буду обозначать его как $Q$:
\begin{myauto}
  \node[state] (q_0) {A};
  \node[state,initial] (s) [above left of=q_0] {S};
  \node[state,accepting] (t) [above right of=q_0] {T};
  \node[state]         (q_2) [below left of=q_0]       {C};
  \node[state]         (q_3) [below right of=q_0] {D};
  \node[state]         (q_1) [left of=q_2] {B};
  \node[state]         (q_4) [right of=q_3]       {E};
  \node[state]         (q_5) [below of=q_1]       {F};
  \node[state]         (q_6) [below of=q_2]       {G};
  \node[state]         (q_7) [below of=q_3]       {H};
  \node[state]         (q_8) [below of=q_4]       {I};
  \node[state]         (q_9) [below right of=q_6]  {J};

  \path[->] (s) edge node [below] {$\epsilon$} (q_0)
            (q_0) edge node [above] {$\epsilon$}   (q_1)
                  edge node [above] {$\epsilon$}   (q_2)
                  edge node [above] {$\epsilon$}   (q_3)
                  edge node [above] {$\epsilon$}   (q_4)
                  edge node [below] {$\epsilon$}   (t)
            (q_1) edge node [left]  {$aa$}         (q_5)
            (q_2) edge node [left]  {$ab$}         (q_6)
            (q_3) edge node [left]  {$abab$}       (q_7)
            (q_4) edge node [left]  {$baba$}       (q_8)
            (q_5) edge node [below] {$\epsilon$}   (q_9)
            (q_6) edge node [below] {$\epsilon$}   (q_9)
            (q_7) edge node [below] {$\epsilon$}   (q_9)
            (q_8) edge node [below] {$\epsilon$}   (q_9)
            (q_9) edge node [left] {$\epsilon$}    (q_0);
\end{myauto}\\

Конкатенируем $PQP$. Тут, кажется, все понятно. Две копии автомата $P$. Конкатенация по $\epsilon$-переходам. Убираем терминальность у всех терминальных, кроме второго $P$.
\end{document}
