\documentclass[12pt]{article}
\usepackage[left=2cm,right=2cm,top=2cm,bottom=2cm,bindingoffset=0cm]{geometry}
\usepackage[utf8x]{inputenc}
\usepackage[english,russian]{babel}
\usepackage{cmap}
\usepackage{amssymb}
\usepackage{amsmath}
\usepackage{url}
\usepackage{pifont}
\usepackage{tikz}
\usepackage{verbatim}
\usepackage{amsthm}

\usetikzlibrary{shapes,arrows}
\usetikzlibrary{positioning,automata}
\tikzset{every state/.style={minimum size=0.2cm},
initial text={}
}


\newenvironment{myauto}[1][3]
{
  \begin{center}
    \begin{tikzpicture}[> = stealth,node distance=#1cm, on grid, very thick]
}
{
    \end{tikzpicture}
  \end{center}
}


\begin{document}
\begin{center} {\LARGE Формальные языки} \end{center}

\begin{center} \Large домашнее задание до 23:59 16.03 \end{center}
\bigskip

\begin{enumerate}
  \item Доказать или опровергнуть свойство регулярных выражений:
  \[
    \forall p, q \text{ --- регулярные выражения}: (p \mid q)^* = p^*(qp^*)^*
  \]

  \item Доказать или опровергнуть свойство регулярных выражений:
  \[
    \forall p, q \text{ --- регулярные выражения}: (p q)^* p = p (q p)^*
  \]

  \begin{proof}

    \[
      \forall p, q \text{ --- регулярные выражения}: (p q)^k p = p (q p)^k
    \]
    Индукция по k:
    \begin{description}
      \item База:
        \[k = 0,   p = p\]
      \item Переход:
        \[(pq)^k p = p (qp)^k \]
        \[ pq ~ (pq)^{k-1}p = pq~p(qp)^{k-1}\]
        Верно, исходя из индукционного предположения:
        \[(pq)^{k-1}p = p(qp)^{k-1}\]
        Очевидно:
        \[pq = pq\]

    \end{description}
    \[\]
  \end{proof}
  \item Доказать или опровергнуть свойство регулярных выражений:
  \[
    \forall p, q \text{ --- регулярные выражения}: (p q)^* = p^* q^*
  \]

  \begin{proof}
    Контрпример, строка разбираемая лишь одним из двух выражений:
    \[
      p
    \]

  \end{proof}

  \item Для регулярного выражения:
   \[ (a \mid b)^+ (aa \mid bb \mid abab \mid baba)^* (a \mid b)^+\]
  Построить эквивалентные:
  \begin{enumerate}
    \item Недетерминированный конечный автомат
    \item Недетерминированный конечный автомат без $\varepsilon$-переходов
    \item Минимальный полный детерминированный конечный автомат
  \end{enumerate}

\end{enumerate}

\end{document}
