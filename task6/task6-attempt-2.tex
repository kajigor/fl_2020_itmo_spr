\documentclass[12pt]{article}
\usepackage[left=2cm,right=2cm,top=2cm,bottom=2cm,bindingoffset=0cm]{geometry}
\usepackage[T2A,T1]{fontenc}
\usepackage[utf8]{inputenc}
\usepackage[english, russian]{babel}
\usepackage{amsmath}
\usepackage[table,xcdraw]{xcolor}
\usepackage{hyperref}

\begin{document}

\bigskip
\begin{enumerate}
 \item Можно ли распознать язык $\{a^kb^mb^pa^m | 0 \leq k < p, 0 \leq m\}$ при помощи алгоритма CYK? 
    Если можно, то привести пример вывода.
    Если нельзя -- объяснить почему.

        Грамматика распознающая язык (этот факт, наверное, следует доказать):
        \[
            S \rightarrow A b B
        \]
        \[
            A \rightarrow a A b~|~\varepsilon 
        \]
        \[
            B \rightarrow b B a~|~b~B~|~\varepsilon
        \]

        Она же в НФК:
        \[
            S \rightarrow A B_1~|~B_1B~|~A C~|~b
        \]
        \[
            C \rightarrow B_1 B
        \]
        \[
            A_1 \rightarrow a
        \]
        \[
            B_1 \rightarrow b
        \]
        \[
            A \rightarrow A_1 A'
        \]
        \[
            A' \rightarrow A B_1~|~b
        \]
        \[
            B \rightarrow B_1 B'~|~B_1 B~|~b
        \]
        \[
            B' \rightarrow B A_1~|~a
        \]
        Таблицу можно сгенерировать с помощью следующего инструмента: \\ \url{http://lxmls.it.pt/2015/cky.html} 

        Грамматика для вставки:

\begin{verbatim}
S -> A B1
S -> B1 B
S -> A C
S -> b
C -> B1 B
A1 -> a
B1 -> b
A -> A1 AN
AN -> A B1
AN -> b
B -> B1 BN
B -> B1 B
B -> b
BN -> B A1
BN -> a
\end{verbatim}

\end{enumerate}

\end{document}
