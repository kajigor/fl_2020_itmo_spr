\documentclass[10pt,twoside,a4paper]{memoir}
\usepackage{graphicx}
\usepackage{forest}
\usepackage{tikz-qtree}
\usepackage[english,russian]{babel}
\begin{document}
Грамматика:

\textsl{\textless S\textgreater} ::= \textit{a};\\

\textsl{Исходная строка:} a

\textbf{Дерево вывода:}

\begin{forest} for tree={edge path={\noexpand\path[\forestoption{edge}] (\forestOve{\forestove{@parent}}{name}.parent anchor) -- +(0,-12pt)-| (\forestove{name}.child anchor)\forestoption{edge label};}}
[[\textsl{\textless S\textgreater}[\textit{a}]], [\$]]
\end{forest}



\begin{center}
\textsl{Дополнительные таблицы:}

\begin{tabular}{ |c||c|c|c| }
\hline
 & Выводится $\varepsilon$ & FIRST таблица & FOLLOW таблица \\
\hline\hline
\textsl{\textless S\textgreater} & false & \{\textit{a}\} & \{\$\}\\
\hline
\end{tabular}

\end{center}     

\begin{center}
\textsl{LL(1) таблица}

\begin{tabular}{ |c||c| }
\hline
 & \textit{a} \\
\hline\hline
\textsl{\textless S\textgreater} & \textit{a}\\
\hline
\end{tabular}

\end{center}
\end{document}