\documentclass[12pt]{article}
\usepackage[left=2cm,right=2cm,top=2cm,bottom=2cm,bindingoffset=0cm]{geometry}
\usepackage[T2A,T1]{fontenc}
\usepackage[utf8]{inputenc}
\usepackage[english, russian]{babel}

\begin{document}

\bigskip
\begin{enumerate}
 \item Можно ли распознать язык $\{a^kb^mb^pa^m | 0 \leq k < p, 0 \leq m\}$ при помощи алгоритма CYK?
Если нельзя -- объяснить.

Докажем, что представленный язык $L$ не распрознаётся алгоритмом $CYK$.

Так как известно, что алгоритм $CYK$ является алгоритмом синтаксического анализа для КС-языков, то
для доказательства требуемого достаточно доказать, что язык не является КС-языком.

Допустим представленный в условии язык является КС-языком.

Тогда, по лемме о накачке для КС-языков, существует такое натуральное число $n$, что для любого слова $w \in L$,
такого что $|w| \geq n$, существуют такие $x, u, y, v, z \in \Sigma^*$, что $xuyvz = w$, причём: 
\begin{enumerate}
 \item $uv \neq \epsilon$,
 \item $|uyv| \leq n$,
 \item а так же слово $q = x u^i y v^i z$ для любого  $i \in Z_{0+}$ так же будет $\in L$.
\end{enumerate}

Утверждается, что для любого предоставленного $n$ можно представить слово $\in L$, но для которого вышеупомянутые свойства не выполняются.

Подобным словом будет: $k=n, p=n+1, m=n$:  
    \[
        a^n b^n b^{n+1} a^n
    \]

Так как $|uyv| \leq n$, то $ uyv $ может иметь вид:

    \begin{enumerate}
            \item $ a^l $ для какого-то l;
               %% Данный случай может случиться в начале или в конце строки. \\
               %% Если в начале, то всегда найдётся достаточно большой $i$, чтобы нарушилось условие $ k < p$.
               %% Если в конце, то при $i=0$ будет нарушено условие про сбалансированность $b$ и $a$ с суперскриптом $m$. 
            \item $ a^l b^k $ для каких-то $l$ и $k$;
            \item $ b^l a^k $ для каких-то $l$ и $k$;
            \item $ b^l $ для какого-то l.
    \end{enumerate}

    Для каждого из этих случаев возможно найти достаточно большой или маленький $i$, что будет обязательно нарушено
какое-то правило языка.

\end{enumerate}

\end{document}
